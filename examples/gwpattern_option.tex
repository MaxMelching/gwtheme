%% Requires compilation with XeLaTeX or LuaLaTeX
\documentclass[
  10pt,
  aspectratio=169,
  xcolor={dvipsnames,table,x11names},  % -- To avoid option clash, this seems to load xcolor before theme and thus avoid clash
]{beamer}


% -- Include theme
\makeatletter
\def\input@path{{../GWtheme/}}
\makeatother
\usetheme{GW}


% -- How To Use GWPattern -----------------------------------------------------
\setbeamertemplate{background}[GWPattern]


% -- The rendering on diagonal lines can look strange if used in combination
% -- with GWPattern. Therefore, having upperleftbg and upperrightbg on same
% -- value usually looks much better. A convenient way to do this (that also
% -- adjusts the pattern height automatically) is to use the following command:
\UpdateGWthemeOptions[%
  % % patternnsquarepertriangle=13,
  % % patternsquaresize=3.6pt,
  % patternnsquarepertriangle=7,
  % patternsquaresize=7pt,
  % 
  % patternnsquarepertriangle=9,
  % patternsquaresize=5pt,
  % patternnsquarecommonshift=2,
  % patternnsquarerelativeshift=-4,
  % 
  patternnsquarepertriangle=13,
  % patternsquaresize=3.6pt,
  % patternnsquarepertriangle=9,
  % patternsquaresize=5pt,
  patternsquaresize=4.5pt,
  patternnsquarecommonshift=-3,
  patternnsquarerelativeshift=6,
  % 
  patterninsetsize=15pt,
  % patternnsquaredownshift=1,
  % levelledbgline=0.1\paperheight,  % Putting as last argument here is important!!
]
\LevelledBGLine{0.1\paperheight}  % Argument = height of colored box


% -- Appearance can also be changed to your liking
\tikzset{
  fillsquare1/.style={
    red,
    line width=1pt,
  },
  fillsquare2/.style={
    blue,
    line width=1pt,
  },
}


% -----------------------------------------------------------------------------


% -- Other Stuff
\UpdateGWBarOptions[templatefile="none", backgroundsignal="none", progressbar=false]  % Deactivate GW signal plotting, still showing headline


\title[Your Short Title]{Your Presentation}
\subtitle{Your subtitle (if there's one)}
% \author{Your Name \and Another Name}
% \author[Your name (short)]{Your Name}
\author[\MakeUppercase{Your name (short)}]{Your Name \and Another Name}  % For use with \insertshortauthor in footline
\institute{Your Faculty/Department}
\date{Date of Presentation}



\begin{document}

{  % -- Brackets required for temporary settings related to titlepage
\setbeamertemplate{background}{}
\begin{frame}[plain, noframenumbering]  % Equivalent to previous line
  \vspace*{-4\baselineskip}%  Adjusts until it fits
  \titlepage
\end{frame}
}


\begin{frame}{Outline}
  \hypersetup{linkcolor=black}
  \tableofcontents[sectionstyle=shaded/show, hideallsubsections]
\end{frame}

\section{Introduction}
  \subsection{Test}

\begin{frame}{Introduction}

\begin{itemize}
  \item Your introduction goes here!
  \item Use \texttt{itemize} to organize your main points.
\end{itemize}

\begin{block}{Examples}
Some examples of commonly used commands and features are included, to help you get started.
\end{block}

\end{frame}

\section{Some \LaTeX{} Examples}

\subsection{Mathematics}

\begin{frame}{Readable Mathematics 42}

Let $X_1, X_2, \ldots, X_n$ be a sequence of independent and identically distributed random variables with $\text{E}[X_i] = \mu$ and $\text{Var}[X_i] = \sigma^2 < \infty$, and let
$$S_n = \frac{X_1 + X_2 + \cdots + X_n}{n}
      = \frac{1}{n}\sum_{i}^{n} X_i$$
denote their mean. Then as $n$ approaches infinity, the random variables $\sqrt{n}(S_n - \mu)$ converge in distribution to a normal $\mathcal{N}(0, \sigma^2)$.

\end{frame}


\end{document}
