\documentclass{beamer}

\usepackage{xcolor}
\definecolor{CaltechOrange}{HTML}{FF6C0C}
\definecolor{CaltechDarkBlue}{HTML}{003B4C}
\definecolor{CaltechDarkGrey}{HTML}{76777B}
\definecolor{CaltechLightGrey}{HTML}{C8C8C8}
\definecolor{CaltechYellowGrey}{HTML}{AAA99F}
\definecolor{CaltechGreenGrey}{HTML}{849895}


\usepackage{tikz}

% \newlength{\themeMPI@footupshift}
% \newlength{\themeMPI@footvertshift}
% \setlength{\themeMPI@footupshift}{0.025\paperheight}
% \setlength{\themeMPI@footvertshift}{0.025\paperheight}

% \pgfdeclareimage[height=0.1\paperheight]{themeMPI@background_logo}{logos/MPI-AEI_wide_E_neg_cmyk}  % White goes best with most colors

\def\testheight{10pt}
\def\testwidth{20pt}

\def\squaresize{4pt}


\usetikzlibrary{decorations.markings}
\usetikzlibrary{calc}

\tikzset{
    fillsquare1/.style={
        % CaltechGreenGrey,
        % CaltechDarkGrey,
        CaltechDarkBlue,  % Slightly darker version of background
        % rotate=45,
        line width=1pt,
        % minimum size=10pt,
        % inner sep=0pt,
        % outer sep=0pt,
        % shape=rectangle,
        % transform shape,
        % fill,
        % opacity=0.5,
        % rotate around={45:(0,0)},
    },
    fillsquarepath1/.style={
        % fg,
        CaltechDarkBlue!90,
        postaction=decorate,
        decoration={
            markings,
            mark = between positions 0 and 1 step 1/8 with {
                % \draw[fillsquare1];
                \draw[fillsquare1] (-3pt, -3pt) rectangle (3pt, 3pt);
                % \node[fillsquare1] {};
            },
        },
    },
    fillsquare2/.style={
        % fg!70!CaltechLightGrey,
        CaltechDarkBlue!70!CaltechLightGrey,
        % rotate=45,
        line width=1pt,
        % opacity=0.5,
    },
    fillsquarepath2/.style={
        % fg,
        CaltechDarkBlue!90,
        postaction=decorate,
        decoration={
            markings,
            mark = between positions 0 and 1 step 1/8 with {
                % \draw[fillsquare2];
                \draw[fillsquare2] (-3pt, -3pt) rectangle (3pt, 3pt);
                % \node[fillsquare2] {};
            },
        },
    },
}


% \def\MeanFootHeight{0.1\paperwidth}
\newlength{\MeanFootHeight}
% \setlength{\MeanFootHeight}{0.08\paperwidth}
\setlength{\MeanFootHeight}{0.1\paperwidth}

% \def\RelShift{-\MeanFootHeight}
\def\RelShift{-0.5\MeanFootHeight}

\begin{document}

\begin{frame}
\begin{tikzpicture}[overlay,remember picture]%
    \clip (current page.south west) -- +(0, 0.85\MeanFootHeight) -- +(\paperwidth, 1.15\MeanFootHeight) -- (current page.south east) -- cycle;
    
    
    % \fill[CaltechDarkBlue] (current page.south west) -- +(0, 0) -- +(\testwidth, \testheight) -- (current page.south east) -- cycle;
    
    \fill[
        CaltechDarkBlue!90
    % ] (current page.south west) -- +(0, 0.08\paperheight) -- +(\paperwidth, 0.16\paperheight) -- (current page.south east) -- cycle;
    ] (current page.south west) -- +(0, \MeanFootHeight+5pt) -- +(\paperwidth, \MeanFootHeight+5pt) -- (current page.south east) -- cycle;

    \draw[fillsquare1,shift=(current page.south)] (0, 0) rectangle (\squaresize, \squaresize) {};
    % \fill[CaltechGreenGrey, rotate=45, shift=(current page.south)] (0, 0) rectangle (\squaresize, \squaresize);


    \draw[
        fillsquare1,
    ] (current page.center);

    
    \draw[
        fillsquarepath1,
    ] (0, 0) -- (2, 2);


    % \draw[
    %     fillsquarepath1,
    % ] (current page.south west)
    %    -- +(0.333\paperwidth/2, 0.1\paperheight)
    %    -- +(0.333\paperwidth, 0)
    %    -- +(0.333\paperwidth + 0.333\paperwidth/2, 0.1\paperheight)
    %    -- +(0.666\paperwidth, 0)
    %    -- +(0.666\paperwidth + 0.333\paperwidth/2, 0.1\paperheight)
    %    -- +(current page.south east);


    % \draw[
    %     fillsquarepath1,
    % ] (current page.south west)
    %    -- ++(\MeanFootHeight, \MeanFootHeight)
    %    -- ++(\MeanFootHeight, -\MeanFootHeight)
    %    -- ++(\MeanFootHeight, \MeanFootHeight)
    %    -- ++(\MeanFootHeight, -\MeanFootHeight)
    %    -- ++(\MeanFootHeight, \MeanFootHeight)
    %    -- ++(\MeanFootHeight, -\MeanFootHeight)
    %    -- ++(\MeanFootHeight, \MeanFootHeight)
    %    -- ++(\MeanFootHeight, -\MeanFootHeight)
    %    -- ++(\MeanFootHeight, \MeanFootHeight)
    %    -- (current page.south east);

    % \begin{scope}[
    %     shift=(current page.south west),
    % ]

    % \draw[
    %     fillsquarepath1,
    %     % shift={(-\MeanFootHeight, 0)},
    % % ] (current page.south west)
    % % ] (0, 0)
    % ] (-0.05\paperwidth, 0)
    %     -- ++(\MeanFootHeight, \MeanFootHeight)
    %     -- ++(\MeanFootHeight, -\MeanFootHeight)
    %     -- ++(\MeanFootHeight, -\MeanFootHeight)
    %     -- ++(\MeanFootHeight, \MeanFootHeight)
    %     -- ++(\MeanFootHeight, \MeanFootHeight)
    %     -- ++(\MeanFootHeight, -\MeanFootHeight)
    %     -- ++(\MeanFootHeight, -\MeanFootHeight)
    %     -- ++(\MeanFootHeight, \MeanFootHeight)
    %     -- ++(\MeanFootHeight, \MeanFootHeight)
    %     % -- (current page.south east);
    %     -- ++(\MeanFootHeight, -\MeanFootHeight)
    %     -- ++(\MeanFootHeight, -\MeanFootHeight);

    % \draw[
    %     fillsquarepath2,
    %     % shift={(0.01\paperwidth, 0)},
    % % ] (current page.south west)
    % % ] (0, 0)
    % ] (-0.175\paperwidth, 0)
    %     -- ++(\MeanFootHeight, -\MeanFootHeight)
    %     -- ++(\MeanFootHeight, \MeanFootHeight)
    %     -- ++(\MeanFootHeight, \MeanFootHeight)
    %     -- ++(\MeanFootHeight, -\MeanFootHeight)
    %     -- ++(\MeanFootHeight, -\MeanFootHeight)
    %     -- ++(\MeanFootHeight, \MeanFootHeight)
    %     -- ++(\MeanFootHeight, \MeanFootHeight)
    %     -- ++(\MeanFootHeight, -\MeanFootHeight)
    %     -- ++(\MeanFootHeight, -\MeanFootHeight)
    %     % -- (current page.south east);
    %     -- ++(\MeanFootHeight, \MeanFootHeight)
    %     -- ++(\MeanFootHeight, \MeanFootHeight);
    % \end{scope}

    \begin{scope}[
        shift=(current page.south west),
    ]

    % \draw[
    %     fillsquarepath1,
    %     % shift={(-\MeanFootHeight, 0)},  % Only required for one
    % ] (0, 0)
    %     -- ++(\MeanFootHeight, \MeanFootHeight)
    %     -- ++(\MeanFootHeight, -\MeanFootHeight)
    %     -- ++(\MeanFootHeight, \MeanFootHeight)
    %     -- ++(\MeanFootHeight, -\MeanFootHeight)
    %     -- ++(\MeanFootHeight, \MeanFootHeight)
    %     -- ++(\MeanFootHeight, -\MeanFootHeight)
    %     -- ++(\MeanFootHeight, \MeanFootHeight)
    %     -- ++(\MeanFootHeight, -\MeanFootHeight)
    %     -- ++(\MeanFootHeight, \MeanFootHeight)
    %     -- ++(\MeanFootHeight, -\MeanFootHeight)
    %     -- ++(\MeanFootHeight, \MeanFootHeight)
    %     -- ++(\MeanFootHeight, -\MeanFootHeight)
    %     -- ++(\MeanFootHeight, \MeanFootHeight)
    %     -- ++(\MeanFootHeight, -\MeanFootHeight);

    % \draw[
    %     fillsquarepath2,
    %     shift={(-\MeanFootHeight, 0)},
    % ] (0, 0)
    %     -- ++(\MeanFootHeight, \MeanFootHeight)
    %     -- ++(\MeanFootHeight, -\MeanFootHeight)
    %     -- ++(\MeanFootHeight, \MeanFootHeight)
    %     -- ++(\MeanFootHeight, -\MeanFootHeight)
    %     -- ++(\MeanFootHeight, \MeanFootHeight)
    %     -- ++(\MeanFootHeight, -\MeanFootHeight)
    %     -- ++(\MeanFootHeight, \MeanFootHeight)
    %     -- ++(\MeanFootHeight, -\MeanFootHeight)
    %     -- ++(\MeanFootHeight, \MeanFootHeight)
    %     -- ++(\MeanFootHeight, -\MeanFootHeight)
    %     -- ++(\MeanFootHeight, \MeanFootHeight)
    %     -- ++(\MeanFootHeight, -\MeanFootHeight)
    %     -- ++(\MeanFootHeight, \MeanFootHeight)
    %     -- ++(\MeanFootHeight, -\MeanFootHeight);
        

    % -- Following more desirable for spacing above
    % \draw[
    %     fillsquarepath2,
    %     shift={(-\MeanFootHeight, 0)},
    % ] (0, 0)
    %     -- ++(\MeanFootHeight, \MeanFootHeight)
    %     -- ++(\MeanFootHeight, -\MeanFootHeight);
    
    % \draw[
    %     fillsquarepath2,
    %     shift={(\MeanFootHeight, 0)},
    % ] (0, 0)
    %     -- ++(\MeanFootHeight, \MeanFootHeight)
    %     -- ++(\MeanFootHeight, -\MeanFootHeight);
    
    % \draw[
    %     fillsquarepath2,
    %     shift={(3\MeanFootHeight, 0)},
    % ] (0, 0)
    %     -- ++(\MeanFootHeight, \MeanFootHeight)
    %     -- ++(\MeanFootHeight, -\MeanFootHeight);
    
    % \draw[
    %     fillsquarepath2,
    %     shift={(5\MeanFootHeight, 0)},
    % ] (0, 0)
    %     -- ++(\MeanFootHeight, \MeanFootHeight)
    %     -- ++(\MeanFootHeight, -\MeanFootHeight);
    
    % \draw[
    %     fillsquarepath2,
    %     shift={(7\MeanFootHeight, 0)},
    % ] (0, 0)
    %     -- ++(\MeanFootHeight, \MeanFootHeight)
    %     -- ++(\MeanFootHeight, -\MeanFootHeight);
    
    % \draw[
    %     fillsquarepath2,
    %     shift={(9\MeanFootHeight, 0)},
    % ] (0, 0)
    %     -- ++(\MeanFootHeight, \MeanFootHeight)
    %     -- ++(\MeanFootHeight, -\MeanFootHeight);

    % -- Adjust end based on your choice of \MeanFootHeight
    % -- -> makes counting of squares and thus setting of spacing much
    % --    easier. And with for loop also less code.
    % \foreach \i in {-1, 1, 3, ..., 9} {
    %     \draw[
    %         fillsquarepath1,
    %         shift={(\i*\MeanFootHeight+\MeanFootHeight, 0)},
    %     ] (0, 0)
    %         -- ++(\MeanFootHeight, \MeanFootHeight)
    %         -- ++(\MeanFootHeight, -\MeanFootHeight);
        
    %     \draw[
    %         fillsquarepath2,
    %         shift={(\i*\MeanFootHeight, 0)},
    %     ] (0, 0)
    %         -- ++(\MeanFootHeight, \MeanFootHeight)
    %         -- ++(\MeanFootHeight, -\MeanFootHeight);
    % }
    \foreach \i in {0, 2, 4, ..., 8} {
        \draw[
            fillsquarepath1,
            shift={(\i*\MeanFootHeight, 0)},
        ] (0, 0)
            -- ++(\MeanFootHeight, \MeanFootHeight)
            -- ++(\MeanFootHeight, -\MeanFootHeight);
        
        \draw[
            fillsquarepath2,
            shift={(\i*\MeanFootHeight + \RelShift, 0)},
        ] (0, 0)
            -- ++(\MeanFootHeight, \MeanFootHeight)
            -- ++(\MeanFootHeight, -\MeanFootHeight);
    }
    \end{scope}

\end{tikzpicture}%

\end{frame}

\end{document}